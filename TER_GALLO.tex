\documentclass[12pt]{article}
\author{Mathieu GALLO}
\title{Travaux d'étude et de recherche}
\setlength{\hoffset}{-70pt}
\setlength{\textwidth}{481pt}
\renewcommand{\baselinestretch}{1.3}

%%%%%%%%%%%%%%%%%%%%%%%%%%%%%%%%%%%%%%
\usepackage{amsmath}
\usepackage{fourier}
\usepackage{amsthm}
\usepackage{enumitem}
\usepackage{titlesec}
\usepackage{color}
\usepackage{pgfornament}
\usepackage[french]{babel}
\usepackage{varioref}
%%%%%%%%%%%%%%%%%%%%%%%%%%%%%%%%%%%%%%
%%%%%%%%% THEOREMES STYLE %%%%%%%%%%%%
\newtheorem*{notation}{Notation}
\newtheorem*{definition}{Défintion}
\newtheorem{lemme}{Lemme}[section]
\newtheorem*{defft}{Définition \& Théorème}
\newtheorem{theoreme}{Théorème}[section]
\newtheorem*{fait}{Fait}
\newtheorem{proposition}{Proposition}[section]
\labelformat{theoreme}{\textbf{théorème~#1}}
\labelformat{lemme}{\textbf{lemme~#1}}
\labelformat{proposition}{\textbf{proposition~#1}}
%%%%%%%%%%%%%%%%%%%%%%%%%%%%%%%%%%%%%%
%%%%%%%%%%%%%%%%%%%%%%%%%%%%%%%%%%%%%%
\newcommand{\HRule}{\rule{\linewidth}{0.5mm}}

\newcommand{\boxXx}[1]{\medskip\noindent\fbox{\begin{minipage}{\textwidth}\vspace{2pt}#1\vspace{2pt}\end{minipage}}\medskip}

\newcommand{\titleBar}{\hspace{0.6em}\titlerule}
\titleformat{\section}  
{\normalfont\large\bfseries}% format
{}% label
{0pt}% sep
{ \arabic{section}.~ }
[\titlerule]


\titleformat{\subsection}  
{\normalfont\normalsize\bfseries}% format
{}
{0pt}% sep
{ \arabic{section}.\arabic{subsection}.~ }

%%%%%%%%%%%%%%%%%%%%%%%%%%%%%%%%%%%%%%




\begin{document}

\begin{titlepage}
	\begin{center}
		
		\textsc{\LARGE Sorbonne Université}\\[2cm]
		
		\textsc{\Large Travaux d'étude et de recherche}\\[1.5cm]
		

		\HRule \\[0.4cm]
		{ \huge \bfseries Autour du théorème de Dvoretzky\\[0.4cm] }
		

		\HRule \\[2cm]
		
		\vspace{1cm}
		\begin{center}
			\textit{"It soon became clear that an outstanding breakthrough in Geometric Functional Analysis had been achieved."}  \\
			\footnotesize Vitali Milman à propos du théorème de Dvoretzky dans \textit{Dvoretzky theorem - thirty years later}
		\end{center}

		
		\vfill 
		
		\begin{minipage}{0.4\textwidth}
			\begin{flushleft} 
				Mathieu GALLO \\
			\end{flushleft}
		\end{minipage}
		\begin{minipage}{0.4\textwidth}
			\begin{flushright} 
				\emph{Enseignant :} Omer Friedland\\
			\end{flushright}
		\end{minipage}
		
		\vspace{5mm}		
		{ date}
		
	\end{center}
\end{titlepage}

\section{Introduction} 
Alexandre Grothendieck en 1956 dans son article "\textit{Sur certaines classes de suites dans les espaces de Banach et le théorème de Dvoretzky-Rogers}", inspiré par le lemme de Dvoretzky-Rogers (1950) propose une conjecture à laquelle Aryeh Dvoretzky répondra positivement en 1961, aboutissant au résultat suivant :
\newline\boxXx{
	\begin{theoreme}[A. Dvoretzky, 1961]
		Il existe une fonction $k:]0,1[\times \mathbb{N}\to \mathbb{N}$, tel que $\forall \varepsilon\in]0,1[$, $k(\varepsilon,n)\stackrel{n\to\infty}{\longrightarrow}\infty$ et pour tout $n\in\mathbb{N}$ et tout compact convexe symétrique $K\subset \mathbb{R}^n$, il existe un sous espace $V\subset  \mathbb{R}^n$ tels que :
		\begin{enumerate}
			\item[(i)] $\dim V =k(\varepsilon,n)$
			\item[(ii)] $\exists r>0$ tel que , $\; r.(V\cap B^n_2) \subset V\cap K \subset (1+\varepsilon)r.(V\cap B^n_2) $
		\end{enumerate}
	\end{theoreme}
}
V. Milman en 1971 donna une nouvelle preuve du théorème de Dvoretzky en utilisant le phénomène de concentration de la mesure, il a de plus amélioré le théorème en donnant une estimation de la dépendance en $n$ pour la dimension de $V$,  $k(\varepsilon,n)\geq c(\varepsilon).\log(n)$.
\newline\boxXx{
	\begin{theoreme}[V. Milman, 1971]\label{thmMil}
		Pour tout $\varepsilon>0$, il existe une constante $c>0$ tel que pour tout $n\in\mathbb{N}$ et pour tout compact convexe symétrique $K\subset \mathbb{R}^n$, il existe un sous espace $V\subset  \mathbb{R}^n$ tels que :
		\begin{enumerate}
			\item[(i)] $\dim V \geq c.\log(n)$
			\item[(ii)] $\exists r>0$ tel que , $\; r.(V\cap B^n_2) \subset V\cap K \subset (1+\varepsilon)r.(V\cap B^n_2) $
		\end{enumerate}
	\end{theoreme}
}

\boxXx{
\begin{definition}
	Soit $(X,||.||_X),(Y,||.||_Y)$ deux espaces normés et $C> 0$, on dit que $X$ s'injecte $C$-continûment dans $Y$, si il existe $T\in\mathcal{L}(X,Y)$ tel que pour tout $x\in X$
	\begin{equation*}
		||x||_X\leq ||Tx||_Y\leq C||x||_X
	\end{equation*}  
\end{definition}
}
\noindent Il existe une reformulation du théorème de Dvoretzky en terme de norme, en utilisant la relation entre un compact convexe symétrique $K$ et la norme $||y||_K=\inf\big\{\lambda \; \; ; \; \;\frac{y}{\lambda}\in K \big\}$.
\newline\boxXx{
	\begin{theoreme}
	Pour tout $\varepsilon>0$ il existe $c>0$ tel que pour tout $n\in \mathbb{N}$ et pour toute normes $||.||$ sur $\mathbb{R}^n$ ,  $l^k_2$ s'injecte $(1+\varepsilon)$-continûment dans $(\mathbb{R}^n,||.||)$ pour un $k\geq c.\log(n)$. 
	\end{theoreme}
}
\begin{notation}
	Pour la suite on utiliseras les notations : 
	\begin{enumerate}
		\item[-] $|.|_n$ la norme euclidienne sur $\mathbb{R}^n$, ou simplement $|.|$ si il n'y a pas d'ambiguïté sur la dimension.
		\item[-] $S^{n-1} = \big\{x\in \mathbb{R}^n\; ;\; |x|=1\big\}$, la $(n-1)$-sphère euclidienne. 
	\end{enumerate}
\end{notation}
\noindent Montrons que ses deux derniers théorèmes sont équivalents.
\begin{enumerate}
	\item[(2)$\Rightarrow$(3)]
	Posons $K=\text{Adh}(B_{||.||}(0,1))=\{x\in\mathbb{R}^n \; | \; ||x||\leq 1 \}$ et appliquons le théorème 2, celui-ci nous procure un sous-espace $V$ de $\mathbb{R}^n$, avec $\dim V := k \geq c.\log(n)$ et $V\cap K$ est $\varepsilon\text{-ecuclidien}$.\\
	Donnons-nous une base orthonormée $\{v_j\}_{1\leq j \leq k}$ de $V$ et posons 
	\begin{equation*}
	\phi :\begin{array}{ccc}
	(V,||.||) & \mapsto &(\mathbb{R}^k,|.|_k) \\
	\sum_{i=1}^{k}x_i v_i & \to & \sum_{i=1}^{k}x_i e_i
	\end{array}
	\end{equation*}
	Soit $v\in V\cap K$ tel que $||v||=1$, comme $K\cap V$ est $\varepsilon$-euclidien on a que 
	\begin{equation*}
	r \leq |v|_n \leq (1+\varepsilon)r
	\end{equation*}
	La borne supérieure est immédiate car  $K\cap V \subset r(1+\varepsilon).(V\cap B^n_2)$, pour la borne inférieure il suffit de remarquer que $(V\cap K)$ est un fermer de $V$ qui contient l'ouvert $r.(V\cap B^n_2)$ de $V$, comme $v$ est dans la frontière de $K\cap V$ il n'est pas dans l'intérieur de $K\cap V$ et donc dans aucun ouvert contenu dans $V\cap K$.\\
	Fixons des coordonnées à $v$ dans la base $\{v_j\}_{1\leq j \leq k}$, $v = \sum_{i=1}^{k}x_iv_i$, on n'a que $|v|_n = \sqrt{\sum_{i=1}^{k}x_i^2}$ et donc :
	\begin{equation*}
	r \leq \sqrt{\sum_{i=1}^{k}x_i^2} \leq (1+\varepsilon)r
	\end{equation*}
	Mais comme $|\phi(v)|_k = \big|\sum_{i=1}^{k}x_i e_i \big| = \sqrt{\sum_{i=1}^{k}x_i^2}$, on a que :
	\begin{equation*}
	r \leq |\phi(v)|_k \leq (1+\varepsilon)r
	\end{equation*}
	Pour tous $x\in  V\backslash\{0\}$ on peut appliquer ce qui précède à $\frac{x}{||x||}$, en utilisant la linéarité de $\phi$ on obtient:
	\begin{equation*}
	r||x||\leq |\phi(x)|_k \leq (1+\varepsilon)r ||x||
	\end{equation*}
	\item[(3)$\Rightarrow$(2)]
	Soit $\varepsilon>0$ , par le théorème 3 il existe $c>0$ tel que pour tous $n\in\mathbb{N}$ il existe un $k>c.\log(n)$ tel que $l_2^k$ s'injecte ($1+\varepsilon$)-continûment dans ($R^n,||.||$) pour n’importe quelle norme $||.||$ sur $\mathbb{R}^n$. Considérons un compact convexe symétrique $K\subset \mathbb{R}^n$ et $||y||=\inf\Big\{\lambda>0\; ;\; \frac{y}{\lambda}\in K\Big\}$, alors $\exists T :l^{k}_2\to(\mathbb{R}^n,||.||)$ linéaire tel que :
	\begin{equation*}
	\forall x \in \mathbb{R}^k \; , \;\; |x|\leq ||Tx||\leq (1+\varepsilon)|x|
	\end{equation*}
	ceci implique immédiatement que $T$ est injective, notons $V=\text{Im}T$, alors la co-restriction a $V$ de $T$ est bijective.
	Soit $y\in \partial(K\cap V)$, c'est-à-dire $||y||=1$, on sait qu'il existe un unique $x\in\mathbb{R}^k$ tel que $Tx=y$, on en déduit donc 
	\begin{equation*}
	|x|\leq 1 \leq (1+\varepsilon)|x|\; \iff\; \frac{1}{1+\varepsilon}\leq|x|\leq 1
	\end{equation*}
	la convexité et la symétrie centrale de $K\cap V$ nous permet de conclure que  :
	\begin{equation*}
		\frac{1}{1+\varepsilon}T(B_2^k)\subset K\cap V \subset T(B_2^k)
	\end{equation*}
\begin{center}\huge\color{red}...\color{black}\end{center}
\end{enumerate}
\newpage

\section{Existence du sous-espace}
\subsection{Mesures de Haar }
\boxXx{
	\begin{defft}[Mesures de Haar]
		Soit $(X,d)$ un espace métrique, $G$ un groupe topologique localement compact qui agit sur $X$ et tel que :
		\begin{equation}\tag{$\star$}
		\forall x,y\in X \; \; \forall g \in G  \; , \; d(gx,gy)=d(x,y)
		\end{equation} 
		alors il existe une unique mesure à un coefficient multiplicatif près, régulière définie sur les boréliens de $X$ qui est invariante sous l'action de $G$, cette mesure est appelée mesure de Haar de $X$ (où $G$ est sous-entendu).
	\end{defft}
}
Considérons $X=S^{n-1}$ avec la distance euclidienne et $X=O(n)$ avec la norme $||M||=\sup_{|x|=1}|Mx|$ alors $G=O(n)$ le groupe des isométries vérifie $(\star)$ pour la multiplication matricielle sur $S^{n-1}$ et $O(n)$.
\begin{notation}
	Par le théorème précédent on peut définir sans ambiguïté $\mu$,$\nu$ les mesures de Haar normalisés respectivement sur $S^{n-1}$ et $O(n)$.
\end{notation}
Montrons quelques propriétés qui seront utiles par la suite.
\newline\boxXx{
	\begin{lemme}
		Soit $f\in C(S^{n-1})$ et $Y=(g_1,...,g_n)$ où les $\{g_i\}_{1\leq i\leq n}$ sont i.i.d suivant une loi normale $\mathcal{N}(0,1)$, alors 
		\begin{equation*}
		\int_{S^{n-1}}fd\mu = \mathbb{E}\Bigg[f\Big(\frac{Y}{|Y|}\Big)\Bigg]
		\end{equation*}
	\end{lemme}
}
\begin{proof}
	Par unicité de la mesure de Haar , il nous suffit de montrer que pour tous $M\in O(n)$ et $f\in C(S^{n-1})$ :
	\begin{equation*}
	\mathbb{E}\Bigg[f\Big(\frac{MY}{|MY|}\Big)\Bigg]=\mathbb{E}\Bigg[f\Big(\frac{Y}{|Y|}\Big)\Bigg]
	\end{equation*}
	\begin{equation*}
	\mathbb{E}\Bigg[f\Big(\frac{MY}{|MY|}\Big)\Bigg] = \int_{\mathbb{R}^n\backslash\{0\}} f\big(\frac{My}{|y|}\big)\exp\Big\{-\frac{1}{2}|y|^2\Big\}dy_1...dy_n=\int_{\mathbb{R}^n\backslash\{0\}} f\big(\frac{y}{|y|}\big) \exp\Big\{-\frac{1}{2}|M^{-1}y|^2\Big\}dy_1...dy_n
	\end{equation*}
	comme $|\det M|=1$ et $|M^{-1}y|=|y|$, on a : 
	\begin{equation*}
	\mathbb{E}\Bigg[f\Big(\frac{MY}{|MY|}\Big)\Bigg] = \mathbb{E}\Bigg[f\Big(\frac{Y}{|Y|}\Big)\Bigg]
	\end{equation*}
\end{proof}

\boxXx{
	\begin{lemme}
		Soit $A\subset S^{n-1}$ un borélien alors pour tous $x\in S^{n-1}$  
		\begin{equation*}
		\nu\Big( T\in O(n) \;;\; Tx\in A\Big) = \mu\big(A \big)
		\end{equation*}
	\end{lemme}
}
\begin{proof}
	Soit $M \in O(n)$ et $x\in S^{n-1}$ alors la mesure définie par
	\begin{equation*}
	\omega_x(A)= \nu\Big( T\in O(n) \;;\; Tx\in A\Big)
	\end{equation*}
	$\omega_x$ vérifie les propriétés suivantes :
	\begin{equation*}
	\omega_x(MA)= \nu\Big( T\in O(n) \;;\; M^{T}Tx\in A\Big)=\nu\Big( T\in O(n) \;;\; Tx\in A\Big)=\omega_x(A)
	\end{equation*}
	\begin{equation*}
	\omega_x(\emptyset)=0
	\end{equation*}
	\begin{align*}
	\omega_x\big(\bigsqcup_{i\in \mathbb{N}}A_i\big)&= \nu\Big( T\in O(n) \;;\; Tx\in \bigsqcup_{i\in \mathbb{N}}A_i\Big)=\nu\Big(\bigsqcup_{i\in\mathbb{N}}\big\{T\in O(n) \;;\;  Tx\in A_i \big\}\Big)\\
	& =\sum_{i\in\mathbb{N}}\nu\Big(T\in O(n) \;;\;  Tx\in A_i \Big)=\sum_{i\in\mathbb{N}}\omega_x(A_i)
	\end{align*}
	L'unicité de la mesure de Haar nous permet de conclure que $\omega_x  = \mu$, en particulier $\omega_x$ ne dépend pas de $x$.
\end{proof}
\subsection{Début de la démonstration du théorème de Dvoretzky}
\begin{notation}
	S'il n'y a pas d'ambiguïté sur la norme $||.||$ de $\mathbb{R}^n$ utilisé on notera :
	\begin{enumerate}
		\item[-] $E=\int_{S^{n-1}}||x||d\mu(x)$
		\item[-] $b$ le plus petit réel tel que $||.||\leq b |.|$
	\end{enumerate} 

\end{notation}
\boxXx{
	\begin{theoreme}[Concentration de la mesure sur la sphère]
		Soit $f:S^{n-1}\to \mathbb{R}$ une fonction Lipschitzienne de constante $L>0$, alors 
		\begin{equation*}
			\mu\Big\{x\in S^{n-1}\; ; \; |f(x)-\mathbb{E}[f]|>\varepsilon\Big\}\leq 2e^{-\frac{\varepsilon^2n}{2L^2}}
		\end{equation*}
	\end{theoreme}
}
\boxXx{
	\begin{definition}
		Soit $(X,d)$ un espace métrique pré-compact et $\theta>0$, on dit que $A\subset X$ est un $\theta$-net si \begin{enumerate}
			\item[(i)] $A$ est de cardinal fini.
			\item[(ii)] $\forall x \in X$ , $\exists y\in A$ tel que $d(x,y)\leq\theta$ 
		\end{enumerate}
	\end{definition}
}
\boxXx{
	\begin{lemme}\label{lns}
		Pour tous $0<\theta<1$ , $V\subset\mathbb{R}^n$ un sous espace de dimension $k>0$, alors il existe un $\theta$-net sur $V\cap S^{n-1}$ de cardinal inférieur à $\big(\frac{3}{\theta}\big)^{k}$.
	\end{lemme}
}
\begin{proof}
	 Notons $B_V(x,r)=\big\{y\in V\; \; ; \; \; |x-y|< r\big\}$ la boule de centre $x\in V$ et de rayon $r\geq 0$, soit $N=\{x_i\}_{i=1,...,m}$ un sous-ensemble de $V\cap S^{n-1}$ maximal pour la propriété : $x,y\in N$ , $|x-y|\geq \theta$, c'est-à-dire pour tous $x\in V\cap S^{n-1}\backslash N$ il existe $i\leq m$ tel que $|x-x_i|<\theta$, donc $N$ est un $\theta$-net et les $\big\{B_V(x_i,\theta/2)\big\}_{i=1,...,m}$ sont donc disjoints deux à deux et toutes contenues dans $B_V(0,1+\frac{\theta}{2})$ d'ou : 
	\begin{equation*}
	m \text{Vol}(B_V(x_1,\frac{\theta}{2}))= \sum_{i=1}^{m}\text{Vol}(B_V(x_i,\frac{\theta}{2}))= \text{Vol}(\cup_{1\leq i \leq m} B_V(x_i,\frac{\theta}{2}))\leq \text{Vol}(B_V(0,1+\frac{\theta}{2}))
	\end{equation*}
	\begin{equation*}
	m\leq \frac{\text{Vol}(B_V(0,1+\frac{\theta}{2}))}{\text{Vol}(B_V(x_1,\frac{\theta}{2}))} 
	\end{equation*}
	Par homogénéité de la mesure de Lebesgue :
	\begin{equation*}
		m\leq\Bigg(\frac{1+\frac{\theta}{2}}{\frac{\theta}{2}}\Bigg)^k=  \Bigg(1+\frac{2}{\theta}\Bigg)^k<\big(\frac{3}{\theta}\big)^k
	\end{equation*}
\end{proof}
\boxXx{
	\begin{lemme}\label{lap}
		Soient $\alpha\in S^{n-1}$ , A un $\theta$-net pour un $1>\theta>0$, alors il existe $(y_i)_{i\in\mathbb{N}}\subset A$ et $(\beta_i)_{i\in\mathbb{N}}\subset \mathbb{R}^{+}$ tel que 
		\begin{equation*}
			\alpha = \sum_{i=0}^{+\infty}y_i \beta_i \text{\hspace{4mm} et \hspace{4mm} $\forall i\in \mathbb{N}$, }\; \beta_i\leq \theta^i
		\end{equation*}
	\end{lemme}
}
\begin{proof}
	Comme $A$ est un $\theta$-net alors il existe $y_0\in A$ tel que $|\alpha-y_0|<\theta$, et donc
	\begin{equation*}
		\alpha = y_0 + \lambda_1 \alpha'
	\end{equation*}
	avec $\lambda_1= |\alpha-y_0|\leq \theta$ et $\alpha' = \frac{\alpha-y_0}{\lambda_1}\in S^{n-1}$, on peut donc itéré le même procédé sur $\alpha'$ et réitéré indéfiniment : 
	\begin{equation*}
	\begin{array}{ccc}
		\alpha = y_0 + \lambda_1 (y_1+\lambda_2\alpha'')=y_0+\lambda_1y_1+\lambda_1\lambda_2\alpha'' &\text{\hspace{4mm} avec \hspace{2mm}}&  \lambda_2\leq \theta, \; y_1\in A \text{\hspace{2mm}et\hspace{2mm}} \alpha''\in S^{n-1}\\
		\vdots&\vdots&\vdots\\
		\alpha = y_0 + \sum_{i=1}^{N}y_i\Big(\prod_{1\leq k\leq i}\lambda_k\Big)+\tilde{\alpha}\prod_{1\leq k\leq N+1}\lambda_k &\text{\hspace{4mm} avec \hspace{2mm}}&  \text{$\forall i\leq N+1$\hspace{1mm}} \lambda_i\leq \theta, y_i\in A \text{\hspace{2mm}et\hspace{2mm}} \tilde{\alpha}\in S^{n-1}\\
		\vdots & \vdots & \vdots
	\end{array}
	\end{equation*} 
	Si l'on pose $S_N = y_0 + \sum_{i=1}^{N}y_i\Big(\prod_{1\leq k\leq i}\lambda_k\Big)$, alors :
	\begin{equation*}
		|\alpha-S_N| \leq |\lambda_1 ... \lambda_N| |\tilde{\alpha}|\leq \theta^{N} \to 0  \text{\hspace{3mm}avec\hspace{2mm}}  N\to \infty
	\end{equation*}
	il ne reste plus qu'as poser $\beta_0=1$ et pour $i>0$, $\beta_i=\prod_{1 \leq k\leq i}\lambda_k\leq \theta^i$ et l'on a :
	\begin{equation*}
		\alpha = \sum_{i=0}^{+\infty} \beta_i y_i
	\end{equation*}
\end{proof}

\boxXx{
	\begin{lemme}\label{lsB}
		$\forall \varepsilon >0$ , il existe $1>\theta>0$ tel que pour tous $n\in \mathbb{N}$ , si l'on a $A$ un $\theta$-net sur $V\cap S^{n-1}$ pour $V\underset{\text{sev}}{\subset}\mathbb{R}^n$ de dimension $k$, $||.||$ une norme sur $\mathbb{R}^{n}$ et $T\in\mathcal{L}(l^k_2,\mathbb{R}^n)$, tel que:
		\begin{equation*}
			\forall  x \in A, \hspace{5mm} (1-\theta)E\leq \big|\big|Tx \big|\big|\leq (1+\theta)E
		\end{equation*}
		alors ,
		\begin{equation*}
			\forall  x \in V, \hspace{5mm} \frac{1}{\sqrt{1+\varepsilon}}E|x|\leq \big|\big|Tx\big|\big|\leq \sqrt{1+\varepsilon}E|x|
		\end{equation*}
		de plus si $\varepsilon\leq \frac{1}{9}$, on peu prendre $\theta=\frac{\varepsilon}{9}$
	\end{lemme}
}
\begin{proof} 
	Soient $1>\theta>0$, $A$ un $\theta$-net sur $S(V)=\big\{x\in V \; ;^|x|=1\big\}$ et $x\in S(V)$ par le \ref{lap},  il existe $(y_i)_{i\in\mathbb{N}}\subset A$ et $(\beta_i)_{i\in\mathbb{N}}\subset \mathbb{R}^{+}$ tel que 
	\begin{equation*}
	x = \sum_{i=0}^{+\infty}y_i \beta_i \text{\hspace{4mm} et \hspace{4mm} $\forall i\in \mathbb{N}$, }\; \beta_i\leq \theta^i
	\end{equation*}
	Notons $T=(a_1,...,a_k)$
	\begin{align*}
		||Tx|| &= \big|\big| T\sum_{i=0}^{+\infty}y_i \beta_i\big|\big|\\
		&=  \big|\big| \sum_{i=0}^{+\infty}\beta_i \sum_{p=1}^{k}y_{i,p}a_p \big|\big|\\
		&\leq \sum_{i=0}^{+\infty}\theta^i ||\sum_{p=1}^{k}y_{i,p}a_p||\\
		&\leq \sum_{i=0}^{+\infty}\theta^i ||Ty_i||\\
		& \leq  \sum_{i=0}^{+\infty}\theta^i (1+\theta)E=\frac{1+\theta}{1-\theta}E
	\end{align*}
	de même :
	\begin{align*}
		||Tx|| &\geq||Ty_0||- ||Tx-Ty_0||\\
		&= E(1-\theta) - ||\sum_{p=1}^{k}a_p\sum_{i=1}^{+\infty}\beta_i y_{i,p}||\\
		&\geq E(1-\theta)- \sum_{i=1}^{+\infty}\theta^i ||Ty_i||\\
		&\geq E\big((1-\theta)- \theta\frac{1+\theta}{1-\theta}\big)= E \frac{1-3\theta}{1-\theta}
	\end{align*}
	Il suffit donc de prendre $\theta$ tel que
	\begin{equation*}
	\begin{array}{cc}
		\sqrt{1+\varepsilon}\geq \frac{1+\theta}{1-\theta}\\
		\frac{1}{\sqrt{1+\varepsilon}}\leq \frac{1-3\theta}{1-\theta}
	\end{array}
	\end{equation*}
	et pour tous $x\in V \backslash\{0\}$ on a 
	\begin{equation*}
		\begin{array}{cc}
			E\frac{1}{\sqrt{1+\varepsilon}}\leq\big|\big|T\frac{x}{|x|}\big|\big|\leq E \sqrt{1+\varepsilon}\\
			E\frac{1}{\sqrt{1+\varepsilon}}|x|\leq||Tx||\leq E|x|\sqrt{1+\varepsilon}
 		\end{array}
	\end{equation*}
	Ce qui fini la première partie de la preuve, dans la suite on suppose $\varepsilon\leq \frac{1}{9}$. On cherche $\theta=:\theta(\varepsilon)\in]0,1[$, tel que $\sqrt{1+\varepsilon}\geq \max\big(\frac{1-\theta}{1-3\theta},\frac{1+\theta}{1-\theta}\big)$, supposons $\theta\leq \frac{1}{3}$ alors
	\begin{equation*}
		\frac{1-\theta}{1-3\theta}-\frac{1+\theta}{1-\theta} = \frac{4\theta^2}{(1-3\theta) (1-\theta)}>0
	\end{equation*}
	Donc $\sqrt{1+\varepsilon}\geq \frac{1-\theta}{1-3\theta}$
	\begin{equation*}
	\begin{array}{ccc}
		1+\varepsilon \geq  \big(\frac{1-\theta}{1-3\theta}\big)^2\\
		(9\varepsilon+8)\theta^2 - 2(3\varepsilon+2)\theta +\varepsilon \geq 0\\
	\end{array}
	\end{equation*}
	les deux racines de ce polynôme sont 0<$\frac{3\varepsilon+2-2\sqrt{1+\varepsilon}}{8+9\varepsilon}<\frac{3\varepsilon+2+2\sqrt{1+\varepsilon}}{8+9\varepsilon}$, on cherche donc un $\theta$ dans $]0,\frac{3\varepsilon+2-2\sqrt{1+\varepsilon}}{8+9\varepsilon}]$. Pour finir 
	\begin{align*}
		\frac{3\varepsilon+2-2\sqrt{1+\varepsilon}}{8+9\varepsilon}&\geq \frac{3\varepsilon+2-2-2\varepsilon}{8+9\varepsilon}=\frac{\varepsilon}{8+9\varepsilon}\\
		&\geq \frac{\varepsilon}{9}
	\end{align*}
	donc pour $\varepsilon\in]0,9^{-1}[$ on peu prendre $\theta(\varepsilon)=\frac{\varepsilon}{9}$.
\end{proof}
\boxXx{
	\begin{theoreme}\label{mtool}
		Pour tous $\varepsilon>0$ il existe $c(\varepsilon)>0$ tel que pour tout $n\in \mathbb{N}$ et pour toute norme $||.||$ sur $\;\mathbb{R}^n$, $l_2^k$ s'injecte $(1+\varepsilon)$-continûment dans $(\mathbb{R}^n,||.||)$, pour $k=: \big[c(\varepsilon).\big(\frac{E}{b}\big)^2n\big]$.
	\end{theoreme}
}
\begin{proof}
	Soit $\varepsilon>0$ , on se donne un $1>\theta>0$ donné par le \ref{lsB} et on note
	\begin{enumerate}
		\item[-] $c(\theta) = \frac{\theta^2}{4\log(\frac{3}{\theta})} $
		\item[-] $V\subset \mathbb{R}^n$ un sous espace avec $\dim V := k = \big[c(\theta)\big(\frac{E}{b}\big)^2n\big] $
		\item[-] $\eta = \frac{\theta E}{b}$
		\item[-] $f(\theta)=2\big(\frac{3}{\theta}\big)^{c(\theta)\big(\frac{E}{b}\big)^2n}e^{-\frac{\eta^2 n}{2}}=2\exp\big(-\frac{\eta^2n}{4}\big)$
	\end{enumerate}
	
	Distinguons deux cas 
	\begin{enumerate}[leftmargin=\labelsep]
		\item[$\circ$] {$f(\theta)\geq1$}\newline on a alors :
		\begin{equation*}
		\frac{\eta^2n}{4}\leq \log(2)
		\end{equation*}
		\begin{equation*}
		k\leq \frac{\eta^2}{4\log(3/\theta)}n\leq \frac{\log(2)}{\log(3/\theta)}<1
		\end{equation*}
		Donc $k=0$, dans ce cas il n'y a rien a montrer.
		\item[$\circ$] $f(\theta)<1$\newline Soit $A$ un $\theta$-net sur $V\cap S^{n-1}$, avec $|A|\leq(\frac{3}{\theta})^k$ nous allons montrer qu'il existe $T\in O(n)$ tel que pour tous $x\in A$ 
		\begin{equation*}
		(1-\theta) E \leq ||Tx|| \leq (1+ \theta)E
		\end{equation*}
		Tous d'abord remarquons ceci :
		\begin{align*}
		1&>f(\theta)\geq 2(\frac{3}{\theta})^ke^{-\frac{\eta^2}{2}n} \\
		&> 2|A|e^{-\frac{\eta^2}{2}n}
		\end{align*}
		On a l'inégalité suivante : 
		\begin{align*}
		\nu\Big(\cap_{x\in A}\big\{T\in O(n)\; ; \;  \big| ||Tx||-E \big|\leq b\eta\big\}\Big)& = 1-\nu\big(\cup_{x\in A}\big\{T\in O(n)\; ; \; \big| ||Tx||-E \big|>b\eta\big\}\big)& \\
		& \geq 1-|A|\nu\big(T\in O(n)\; ; \; \big| ||Ty||-E \big|>b\eta\big)& \text{pour un $y\in A$}\\
		& \geq 1 - |A|\mu\Big(y\in S^{n-1}\; ;\; \big| ||y||-E \big|>b\eta\Big)
		\end{align*} 
		En appliquant la concentration de la mesure
		\begin{equation*}
		\nu\Big(\cap_{x\in A}\big\{T\in O(n)\; ; \;  \big| ||Tx||-E \big|\leq b\eta\big\}\Big)\geq 1 - |A|2e^{-\frac{\eta^2n}{2}} >0
		\end{equation*}	
		Il existe donc $T\in O(n)$ tel que pour tous $x\in A$ on ait $\big|||Tx||-E \big|\leq b \eta$, c'est à dire 
		\begin{equation*}
		E(1-\theta)=E-b\eta\leq ||Tx||\leq E+b\eta=E(1+\theta)
		\end{equation*}
		Par le \ref{lsB} pour tous $x\in V$
		\begin{equation*}
		\frac{1}{\sqrt{1+\varepsilon}}|x| E\leq ||Tx||\leq \sqrt{1+\varepsilon}|x| E
		\end{equation*}
	\end{enumerate}
	et pour $\varepsilon<9^{-1}$ on peut prendre $\theta(\varepsilon)=\frac{\varepsilon}{9}$ et donc $c(\varepsilon)=\frac{\varepsilon^2}{4\times 81\log(\frac{3\times9}{\varepsilon})} $.\\ 
\end{proof}




\newpage
\section{Minoration de la dimension du sous-espace }
\subsection{Invariance pour des compositions par des applications linéaire}
Pour simplifier un peu les calculs nous allons montrer que l'on peut se restreindre aux normes qui vérifies $||.||\leq |.|$ et qui ont de plus la propriété d'être l'ellipsoïde de John, c'est à dire l'ellipsoïde de volume maximal incluse dans $K$, nous donnons la définition d'une ellipsoïde et un théorème de Fritz John sur l'unicité de l'ellipsoïde de volume maxime qui seras admis.  
\newline\boxXx{
	\begin{definition}
		Un ellipsoïde de $\mathbb{R}^n$ est l'image de la boule unité euclidienne par un élément de $GL(n)$. 
	\end{definition}
}
\boxXx{
	\begin{theoreme}[Ellipsoïde de John]
		Tous compact convexe symétrique d'intérieur non vide contient un unique ellipsoïde de volume maximale.
	\end{theoreme}
}

\begin{center}\huge\color{red}...\color{black}\end{center}
\subsection{Estimation de $E$}
Par la suite on fixe $||.||$ une norme sur $\mathbb{R}^n$, $K=\text{Adh}\big(B_{||.||}\big)$ tel que $B_2^n$ soit l'ellipsoïde de volume maximale incluse dans $K$, on a donc $b=1$. Dans cette partie nous allons donner une estimation de $E$.

Pour estimer $E$ nous aurons besoin d'une minoration de $\mathbb{E}\Big[\max_{1\leq i \leq N}g_i\Big]$ pour des $\{g_i\}$ i.i.d suivant $\mathcal{N}(0,1)$, nous démontrons une telle borne dans le lemme suivant.  
\newline\boxXx{
	\begin{lemme}
		il existe $c>0$ tel que $\forall N>1$ et $\{g_i\}_{1\leq i \leq N}$ des variables aléatoire i.i.d suivant une loi $\mathcal{N}(0,1)$ on ait :
		\begin{equation*}
			c \sqrt{\log N} \leq \mathbb{E}\big[\max_{1 \leq i\leq \tilde{N}}|g_i|\big]
		\end{equation*}
		où $\tilde{N} = \Bigg[\frac{\sqrt{N}}{16\sqrt{2}}\Bigg]$
	\end{lemme}
}
\begin{proof}
	Commençons par montrer que pour $n>1$, $\mathbb{P}\big(|g_1|> \sqrt{\log n}\big) \geq \frac{1}{n}$, on a :
	\begin{equation*}
		\mathbb{P}\big(|g_1|> \sqrt{\log n}\big) = 2 \int_{\sqrt{\log n}}^{+\infty}e^{-\frac{x^2}{2}}dx\geq \int_{\sqrt{\log n}}^{+\infty}e^{-\frac{x^2}{2}}(1+\frac{1}{x^2})dx  \qquad \text{ pour $x>\sqrt{\log(2)}>1$}
	\end{equation*}
	\begin{equation*}
		\int_{\sqrt{\log n}}^{+\infty}e^{-\frac{x^2}{2}}(1+\frac{1}{x^2})dx = \Big[-\frac{ e^{-\frac{t^2}{2}}}{t}\Big]_{\sqrt{\log n}}^{+\infty}=\frac{1}{\sqrt{n\log n}}>\frac{1}{n}
	\end{equation*}
	On à $\mathbb{P}\big(|g_1|> \sqrt{\log \sqrt{N}}\big)\geq \frac{1}{\sqrt{N}}$, et donc 
	\begin{equation*}
		 \mathbb{P}\Big(\max_{1 \leq i\leq \tilde{N}}|g_i|\leq \sqrt{\log \sqrt{N}}\Big) = \mathbb{P}\Big(|g_1|\leq \sqrt{\log \sqrt{N}}\Big)^{\tilde{N}}=\Bigg(1-\mathbb{P}\Big(|g_1|> \sqrt{\log \sqrt{N}}\Big)\Bigg)^{\tilde{N}}
	\end{equation*}
	\begin{equation*}
		\mathbb{P}\Big(\max_{1 \leq i\leq \tilde{N}}|g_i|\leq \sqrt{\log \sqrt{N}}\Big)\leq \Bigg(1-\frac{1}{\sqrt{N}}\Bigg)^{\tilde{N}}\leq e^{-\frac{\tilde{N}}{\sqrt{N}}} \leq e^{-\frac{1}{16\sqrt{2}}}
	\end{equation*}
	Ce qui équivaut a 
	\begin{equation*}
		\mathbb{P}\Big(\max_{1 \leq i\leq \tilde{N}}|g_i|> \sqrt{\log \sqrt{N}}\Big)\geq 1- e^{-\frac{1}{16\sqrt{2}}}
	\end{equation*}
	Par l'inégalité de Markov on a finalement :
	\begin{equation*}
		\mathbb{E}\big[\max_{1 \leq i\leq \tilde{N}}|g_i|\big]\geq \mathbb{P}\Big(\max_{1 \leq i\leq \tilde{N}}|g_i|> \sqrt{\log \sqrt{N}}\Big) \sqrt{\log \sqrt{N}}\geq \frac{1- e^{-\frac{1}{16\sqrt{2}}}}{\sqrt{2}}\sqrt{\log N}
	\end{equation*}
	avec $c =: \frac{1- e^{-\frac{1}{16\sqrt{2}}}}{\sqrt{2}}\approx 0.3235 $ 
\end{proof}



\boxXx{
	\begin{lemme}[Dvoretzky-Rogers]
 		Il existe une base orthonormée $\{x_i\}_{i=1,...,n}$ tel que $\forall 1\leq i\leq n$
		\begin{equation*}
			e^{-1}\big(1-\frac{i-1}{n}\big)\leq ||x_i||\leq 1 
		\end{equation*}
	\end{lemme}
}
\begin{proof}
	$S^{n-1}$ est compact et $||.||$ continue, on peux donc prendre un $x_1\in S^{n-1}$ qui maximise $||.||$ c'est à dire $||x_1||=1$, supposons que l'on ai $x_1,...,x_{k-1}$ avec $k\leq n$ tel que pour tous $1\leq i\leq k-1$ , $x_i$ maximise $||.||$ sur $S^{n-1}\backslash \text{Vect}(x_1,...,x_{k-1})\neq \emptyset$ car les $\{x_i\}_{i=1,...,k-1}$ sont orthogonaux deux à deux. On peut donc répéter le procéder pour trouver $x_{k}$ qui maximise $S^{n-1}\backslash \text{Vect}(x_1,...,x_{k-1})$, par récurrence on peut donc avoir $n$ vecteurs avec ses propriétés. Fixons $1\leq k \leq n$, $a,b\in\mathbb{R}^{*}$ et définissons :
	\begin{equation*}
		\mathcal{E} = \Big\{\sum_{i=1}^{n}a_ix_i\; ; \; \sum_{i=1}^{k-1}\big(\frac{a_i}{a}\big)^2+ \sum_{i=k}^{n}\big(\frac{b_i}{b}\big)^2\leq 1 \Big\}
	\end{equation*}
	Supposons $\sum_{i=1}^{n}a_ix_i\in \mathcal{E}$, alors $\sum_{i=1}^{k-1}a_ix_i\in aB_2^n$ et donc $||\sum_{i=1}^{k-1}a_ix_i||\leq a$.
	Si $x\in \text{Vect}(x_k,...,x_n)\cap B^n_2$ on a $||x||\leq ||x_k||$ par construction, et donc $\sum_{i=k}^{n}a_ix_i\in bB_2^n \; \Rightarrow \; ||\sum_{i=k}^{n}a_ix_i ||\leq b||x_k||$ , ce qui nous donne la majoration suivante 
	\begin{equation*}
		||\sum_{i=1}^{n}a_ix_i||\leq ||\sum_{i=1}^{k-1}a_ix_i||+||\sum_{i=k}^{n}a_ix_i||\leq a + b||x_k||
	\end{equation*}
	Posons $\phi\in GL(n)$ définit par $\phi(\sum_{i=1}^{n}a_ix_i)=\sum_{i=1}^{k-1}aa_ix_i+\sum_{i=k}^{n}ba_ix_i$ on a $\phi = \text{diag}\big(\overbrace{a,...,a}^{(k-1)\times},\overbrace{b,...,b}^{(n-k+1) \times}\big)$ et donc $\det \phi = a^{k-1}b^{n-k+1}$ d'où :
	\begin{equation*}
		\int_{\mathcal{E}} dx_1...dx_n = \int_{B_2^n} \det \phi dx_1...dx_n= a^{k-1}b^{n-k-1}\int_{B_2^n}dx_1...dx_n
	\end{equation*}
	On prend $a+b||x_k||= 1$ de sorte que $\mathcal{E}\subset K$, comme $B_2^n$ est l'ellipsoïde de volume maximale inclue dans $K$, on a que  
	\begin{equation*}
		1\geq\frac{\int_{\mathcal{E}} dx_1...dx_n}{\int_{B_2^n}dx_1...dx_n}=a^{k-1}b^{n-k+1}
	\end{equation*}
	Fixons donc pour $k\geq 2$ , $b=\frac{1-a}{||x_k||}$ et $a=\frac{k-1}{n}$, en remplaçant dans l'inégalité on obtient :
	\begin{equation*}
		1\geq a^{k-1} \Big(\frac{1-a}{||x_k||}\Big)^{n-k+1} \; \iff \; ||x_k||\geq a^{\frac{k-1}{n-k+1}}(1-a) = \Big(\frac{k-1}{n}\Big)^{\frac{k-1}{n-k+1}}\Big(1-\frac{k-1}{n}\Big)
	\end{equation*}
	et $\log a^{\frac{k-1}{n-k+1}}= \frac{k-1}{n-k+1}\log\Big(\frac{k-1}{n}\Big)$\color{red} $>-1$.\color{black}

\end{proof}

\boxXx{
	\begin{proposition}[Estimation de $E$]\label{esE}
		Il existe $c>0$ tel que $E \geq c \sqrt{\frac{\log n}{n}}$.
	\end{proposition}
}
\begin{proof}
	Par le lemme de Dvoretzky-Rogers il existe une base orthonormé $x_1,...,x_n$ tel que pour $1\leq i \leq \tilde{n}=:\big[\frac{\sqrt{n}}{16\sqrt{2}}\big](\leq n)$, $||x_i||\geq e^{-1}\Big(1-\frac{\tilde{n} -1}{n}\Big)\geq e^{-1}\Big(1+\frac{1}{n}-\frac{1}{16\sqrt{2n}}\Big)\geq e^{-1}\big(1-\frac{1}{16\sqrt(2)}\big) $, posons donc $C=:e^{-1}\big(1-\frac{1}{16\sqrt(2)}\big)\approx0,3516$. Comme $\mu$ est invariante par composition par une transformation orthogonale on a que  
	\begin{equation*}
		E=:\int_{S^{n-1}} ||\sum_{i=1}^{n}a_ix_i||d\mu(a)= \int_{S^{n-1}} ||\sum_{i=1}^{n-1}a_ix_i-a_nx_n||d\mu(a)
	\end{equation*}
	et donc 
	\begin{align*}
		E&=\frac{1}{2}\int_{S^{n-1}} ||\sum_{i=1}^{n}a_ix_i||d\mu(a)+ \frac{1}{2}\int_{S^{n-1}} ||\sum_{i=1}^{n-1}a_ix_i-a_nx_n||d\mu(a)\\
		&\geq\frac{1}{2}\int_{S^{n-1}} 2\max\Big\{||\sum_{i=1}^{n-1}a_ix_i||,||a_nx_n||\Big\}d\mu(a)\geq ...\geq \int_{S^{n-1}} \max_{1\leq i \leq n}\Big\{|a_i|\;||x_i||\Big\}d\mu(a)\\
		&\geq \int_{S^{n-1}} \max_{1\leq i \leq \tilde{n}}\Big\{|a_i|\;||x_i||\Big\}d\mu(a) \geq C\int_{S^{n-1}} \max_{1\leq i \leq \tilde{n}}|a_i| d\mu(a)	
	\end{align*}
	Soit $(g_1,...,g_n)$ , des variables aléatoire i.i.d de loi $\mathcal{N}(0,1)$ alors 
	\begin{equation*}
		\int_{S^{n-1}} \max_{1\leq i \leq \tilde{n}}|a_i| d\mu(a) =\mathbb{E}\Big[\big(\sum_{i=1}^{n}g_i^2\big)^{-\frac{1}{2}} \max_{1\leq i \leq \tilde{n}}|g_i|\Big]
	\end{equation*}
	\begin{lemme}
		$\big(\sum_{i=1}^{n}g_i^2\big)^{-\frac{1}{2}}(g_1,...,g_n)$ et $\big(\sum_{i=1}^{n}g_i^2\big)^{\frac{1}{2}}$ sont indépendants.
	\end{lemme}
	\begin{proof}[\color{red}Démonstration du lemme]
	\end{proof}
	\color{black}
	Par le lemme on à donc 
	\begin{equation*}
		\mathbb{E}\Big[\big(\sum_{i=1}^{n}g_i^2\big)^{-\frac{1}{2}} \max_{1\leq i \leq \tilde{n}}|g_i|\Big] . \mathbb{E}\Big[\big(\sum_{i=1}^{n}g_i^2\big)^{\frac{1}{2}}\Big] = \mathbb{E}\big[\max_{1\leq i \leq \tilde{n}}|g_i|\big]
	\end{equation*}
	la fonction racine carré est concave, par l'inégalité de Jensen on a donc :
	\begin{equation*}
		\mathbb{E}\big[\big(\sum_{i=1}^{n}g_i^2\big)^{\frac{1}{2}}\big]\leq \mathbb{E}\big[\sum_{i=1}^{n}g_i^2\big]^{\frac{1}{2}}= \sqrt{n} \mathbb{E}[g_1^2]^{\frac{1}{2}}=\sqrt{n} 
	\end{equation*}

	Et finalement par le lemme 4, il existe $K>0$ tel que :
	\begin{equation*}
		E\geq \frac{C}{\sqrt{n}} \mathbb{E}\big[\max_{1\leq i \leq \tilde{n}}|g_i|\big]\geq KC\sqrt{\frac{\log n}{n}}
	\end{equation*}
	Pour finir il suffit de poser $c=: KC \approx 0.1137$
\end{proof}

On peut donc réunir les résultats \ref{esE} et \ref{mtool} pour obtenir :
\begin{equation*}
	k \geq \big[c(\varepsilon)\log(n)\big]
\end{equation*}
avec $c(\varepsilon)= c_0 \frac{\varepsilon^2}{\log(\frac{21}{\varepsilon})}$ pour $\varepsilon\leq \frac{1}{9}$.
\newpage

\section*{Sources}

- Euclidean sections of convex bodies , Gideon Schechtman (2008)

- Sur certains classes de suites dans les espaces de Banach, et le théorème de Dvoretzky-Rogers, Alexandre Grothendieck (1956) 

-Dvoretzky theorem - thirty years later , Vitali Milman (1992)


\end{document}